%%%%%%%%%%%%%%%%%%%%%%%%%%%%%%%%%%%%%%%%%
% Beamer Presentation
% LaTeX Template
% Version 1.0 (10/11/12)
%
% This template has been downloaded from:
% http://www.LaTeXTemplates.com
%
% License:
% CC BY-NC-SA 3.0 (http://creativecommons.org/licenses/by-nc-sa/3.0/)
%
%%%%%%%%%%%%%%%%%%%%%%%%%%%%%%%%%%%%%%%%%

%----------------------------------------------------------------------------------------
%	PACKAGES AND THEMES
%---------------------------------------------------------------------------------------

\documentclass{beamer}
\usepackage[spanish]{babel}
 \usepackage{algpseudocode}
\usepackage[utf8]{inputenc}
\mode<presentation> {

% The Beamer class comes with a number of default slide themes
% which change the colors and layouts of slides. Below this is a list
% of all the themes, uncomment each in turn to see what they look like.

%\usetheme{default}
%\usetheme{AnnArbor}
%\usetheme{Antibes}
%\usetheme{Bergen}
%\usetheme{Berkeley}
%\usetheme{Berlin}
%\usetheme{Boadilla}
%\usetheme{CambridgeUS}
%\usetheme{Copenhagen}
%\usetheme{Darmstadt}
%\usetheme{Dresden}
%\usetheme{Frankfurt}
%\usetheme{Goettingen}
%\usetheme{Hannover}
%\usetheme{Ilmenau}
%\usetheme{JuanLesPins}
%\usetheme{Luebeck}
%\usetheme{Madrid}
%\usetheme{Malmoe}
%\usetheme{Marburg}
%\usetheme{Montpellier}
\usetheme{PaloAlto}
%\usetheme{Pittsburgh}
%\usetheme{Rochester}
%\usetheme{Singapore}
%\usetheme{Szeged}
%\usetheme{Warsaw}

% As well as themes, the Beamer class has a number of color themes
% for any slide theme. Uncomment each of these in turn to see how it
% changes the colors of your current slide theme.

%\usecolortheme{albatross}
%\usecolortheme{beaver}
%\usecolortheme{beetle}
%\usecolortheme{crane}
%\usecolortheme{dolphin}
%\usecolortheme{dove}
%\usecolortheme{fly}
%\usecolortheme{lily}
%\usecolortheme{orchid}
%\usecolortheme{rose}
%\usecolortheme{seagull}
%\usecolortheme{seahorse}
%\usecolortheme{whale}
%\usecolortheme{wolverine}

%\setbeamertemplate{footline} % To remove the footer line in all slides uncomment this line
%\setbeamertemplate{footline}[page number] % To replace the footer line in all slides with a simple slide count uncomment this line

%\setbeamertemplate{navigation symbols}{} % To remove the navigation symbols from the bottom of all slides uncomment this line
}

\usepackage{graphicx} % Allows including images
\usepackage{booktabs} % Allows the use of \toprule, \midrule and \bottomrule in tables

%----------------------------------------------------------------------------------------
%	TITLE PAGE
%----------------------------------------------------------------------------------------

\title[Practica 3]{Minimizando el número de visitas al proveedor} % The short title appears at the bottom of every slide, the full title is only on the title page

\author{Algorítmica} % Your name
\institute[UGR] % Your institution as it will appear on the bottom of every slide, may be shorthand to save space
{
Universidad de Granada \\ % Your institution for the title page
\medskip

}
\date{\today} % Date, can be changed to a custom date

\begin{document}

\begin{frame}
\titlepage % Print the title page as the first slide
\end{frame}

\begin{frame}
\frametitle{Índice} % Table of contents slide, comment this block out to remove it
\tableofcontents % Throughout your presentation, if you choose to use \section{} and \subsection{} commands, these will automatically be printed on this slide as an overview of your presentation
\end{frame}

%----------------------------------------------------------------------------------------
%	PRESENTATION SLIDES
%----------------------------------------------------------------------------------------

\section{Introducción }
\begin{frame}
	\frametitle{Introducción}
	\begin{itemize}
		\item El objetivo de esta práctica es diseñar un algoritmo Greedy, que resuelva de manera óptima uno de los cinco problemas de la práctica y demostrar que dicho algoritmo encuentra siempre la solución óptima.
	\end{itemize}
\end{frame}


%------------------------------------------------
\section{Ejercicio} 
\begin{frame}
	\frametitle{Enunciado del ejercicio}
	Un granjero necesita disponer siempre de fertilizante. El granjero consume el fertilizante en \textbf{R} días, antes de que esto ocurra debe acudir a abastecerse. Para esto el granjero se desplaza a la tienda del pueblo, de la que conoce el horario. El granjero desea minimizar el número de desplazamientos al pueblo para abastecerse.
	
\end{frame}

%------------------------------------------------
\section{Diseño del algoritmo} 
\begin{frame}
	\frametitle{Diseño del algoritmo}
	\begin{itemize}
		\item \textbf{Conjunto de candidatos}: Conjunto de días en que la tienda permanece abierta. (Conjunto \textbf{C})
		\item \textbf{Conjunto de seleccionados}: Días elegidos para ir a la tienda. (Conjunto \textbf{S})
		\item \textbf{Función solución}: Cuando el conjunto de candidatos esté vacío y aún tengamos fertilizante.
		\item \textbf{Función factibilidad:} No habrá una solución, si dentro del intervalo R no hay un candidato que elegir.
		\item \textbf{Función selección}: Se seleccionará el día más lejado de los posibles candidatos.
		\item \textbf{Función objetivo}: Lista con los días que el granjero irá a la tienda.		
	\end{itemize}
	
\end{frame}

\section{Pseudocodigo}
\begin{frame}
	\frametitle{Pseudocódigo I}
	Selección

	\begin{algorithmic}
	\Require Conjunto de candidatos C
	\State {x=0}
	\For {i = 1 to len(C)}
		\If{ c\_i - c\_dia\_actual menor que R }
 	   		\State{x = c\_i}
		\EndIf
	\EndFor  
	
	\Return x	
	\end{algorithmic}	
\end{frame}

\begin{frame}
	\frametitle{Pseudocódigo II}

	Factibilidad
	\begin{algorithmic}

	\Require Candidato c
	
\If{ c\_dia\_actual == c }
 	   		
 	   		\Return False
 	   	\Else 
 	   	
 	   		\Return True 	   	
 	   	\EndIf
 	 
  	
		\end{algorithmic}	
\end{frame}

\begin{frame}
	\frametitle{Pseudocódigo III}
	\begin{algorithmic}
	\Require Conjunto de candidatos C
	
 	\State{S = 0}

	\While {S no sea una solución y C != 0 }
 	   \State{x = seleccion (C) }
 	   \State{C = C - {x}}
  	   \If{factible({x})}
  	   		\State{S = S + {x}}
  	   	\EndIf
	\EndWhile
	\end{algorithmic}	
	
\end{frame}

%------------------------------------------------
\section{Demostración} 
\begin{frame}
	\frametitle{Demostración}
	Harémos la demostración por reducción al absurdo:
	
	Sea $ L = g_0 < g_1 < ... < g_p $ el conjunto de días seleccionados por el algoritmo Greedy que \textbf{NO} es óptimo.\\
	
	Sea $ L_{op} = f_0 < f_1 < ... < f_p $ una de la soluciones óptimas del problema.\\
	
	Sea $r$ el máximo valor posible hasta donde L y $L_{op}$ coinciden, es decir, $f_0=g:0, f_1=g_1,...,f_r = g_r$.\\
	Entonces, $g(r+1) > f(r+1)$ \\
	Por lo que $g_0 < ... < g_r < g(r+1) < f(r+2) < f_q $ es otra solución al problema. 
	
	Si f llega un momento que se separa de g querrá decir que el número de días será mayor o igual.\\
	Por lo que alcanzamos una contradicción, ya que $r$  no sería el máximo valor donde $L$ y $L_{op}$ siguen siendo iguales.
\end{frame}


\end{document} 
