%%%%%%%%%%%%%%%%%%%%%%%%%%%%%%%%%%%%%%%%%
% Short Sectioned Assignment LaTeX Template Version 1.0 (5/5/12)
% This template has been downloaded from: http://www.LaTeXTemplates.com
% Original author:  Frits Wenneker (http://www.howtotex.com)
% License: CC BY-NC-SA 3.0 (http://creativecommons.org/licenses/by-nc-sa/3.0/)
%%%%%%%%%%%%%%%%%%%%%%%%%%%%%%%%%%%%%%%%%

%----------------------------------------------------------------------------------------
%	PACKAGES AND OTHER DOCUMENT CONFIGURATIONS
%----------------------------------------------------------------------------------------

\documentclass[paper=a4, fontsize=11pt]{scrartcl} % A4 paper and 11pt font size

% ---- Entrada y salida de texto -----

\usepackage{hyperref}
\usepackage{listings}
\usepackage{color}
%AÑADIDO DE LA PÁGINA http://stackoverflow.com/questions/3175105/how-to-insert-code-into-a-latex-doc
\definecolor{dkgreen}{rgb}{0,0.6,0}
\definecolor{gray}{rgb}{0.5,0.5,0.5}
\definecolor{mauve}{rgb}{0.58,0,0.82}

\lstset{frame=tb,
	language=Python,
	aboveskip=3mm,
	belowskip=3mm,
	showstringspaces=false,
	columns=flexible,
	basicstyle={\small\ttfamily},
	numbers=none,
	numberstyle=\tiny\color{gray},
	keywordstyle=\color{blue},
	commentstyle=\color{dkgreen},
	stringstyle=\color{mauve},
	breaklines=true,
	breakatwhitespace=true,
	tabsize=3
}
%%%%%%%%%%%%%%%%%%%%%%%%%%%%%%%%%%%%%%%%%%%%%%%%%%%%%%%
\usepackage{varioref}
\usepackage[T1]{fontenc} % Use 8-bit encoding that has 256 glyphs
\usepackage[utf8]{inputenc}
%\usepackage{fourier} % Use the Adobe Utopia font for the document - comment this line to return to the LaTeX default

% ---- Idioma --------

\usepackage[spanish, es-tabla]{babel} % Selecciona el español para palabras introducidas automáticamente, p.ej. "septiembre" en la fecha y especifica que se use la palabra Tabla en vez de Cuadro

% ---- Otros paquetes ----

\usepackage{amsmath,amsfonts,amsthm} % Math packages
%\usepackage{graphics,graphicx, floatrow} %para incluir imágenes y notas en las imágenes
\usepackage{graphics,graphicx, float} %para incluir imágenes y colocarlas

% Para hacer tablas comlejas
%\usepackage{multirow}
%\usepackage{threeparttable}

%\usepackage{sectsty} % Allows customizing section commands
%\allsectionsfont{\centering \normalfont\scshape} % Make all sections centered, the default font and small caps

\usepackage{fancyhdr} % Custom headers and footers
\pagestyle{fancyplain} % Makes all pages in the document conform to the custom headers and footers
\fancyhead{} % No page header - if you want one, create it in the same way as the footers below
\fancyfoot[L]{} % Empty left footer
\fancyfoot[C]{} % Empty center footer
\fancyfoot[R]{\thepage} % Page numbering for right footer
\renewcommand{\headrulewidth}{0pt} % Remove header underlines
\renewcommand{\footrulewidth}{0pt} % Remove footer underlines
\setlength{\headheight}{13.6pt} % Customize the height of the header

\numberwithin{equation}{section} % Number equations within sections (i.e. 1.1, 1.2, 2.1, 2.2 instead of 1, 2, 3, 4)
\numberwithin{figure}{section} % Number figures within sections (i.e. 1.1, 1.2, 2.1, 2.2 instead of 1, 2, 3, 4)
\numberwithin{table}{section} % Number tables within sections (i.e. 1.1, 1.2, 2.1, 2.2 instead of 1, 2, 3, 4)

\setlength\parindent{0pt} % Removes all indentation from paragraphs - comment this line for an assignment with lots of text

\newcommand{\horrule}[1]{\rule{\linewidth}{#1}} % Create horizontal rule command with 1 argument of height


\renewcommand{\reftextbefore}
	{en la  \reftextvario{página que precede a esta}{página anterior}}
\renewcommand{\reftextafter}
	{en la \reftextvario{siguiente}{siguiente} página}
\renewcommand{\reftextfacebefore}
	{en la  \reftextvario{anterior}{anterior} página}
\renewcommand{\reftextfaceafter}
	{en la \reftextvario{siguiente}{siguiente}{página}}

 \usepackage{algpseudocode}
%----------------------------------------------------------------------------------------
%	TÍTULO Y DATOS DEL ALUMNO
%----------------------------------------------------------------------------------------

\title{	
\normalfont \normalsize 
\textsc{{\bf Algorítmica (2015-2016)} \\ Grado en Ingeniería Informática \\ Universidad de Granada} \\ [25pt] % Your university, school and/or department name(s)
\horrule{0.5pt} \\[0.4cm] % Thin top horizontal rule
\huge Práctica 3-Primera Parte: Minimizando el número de visitas al proveedor \\ % The assignment title
\horrule{2pt} \\[0.5cm] % Thick bottom horizontal rule
}

\author{Francisco Carrillo Pérez,Borja Cañavate Bordons, \\Miguel Porcel Jiménez,Jose Manuel Rejón Santiago,Jose Arcos Aneas} % Nombre y apellidos

\date{\normalsize\today} % Incluye la fecha actual

%----------------------------------------------------------------------------------------
% DOCUMENTO
%----------------------------------------------------------------------------------------

\begin{document}

\maketitle % Muestra el Título

\newpage %inserta un salto de página

\tableofcontents % para generar el índice de contenidos

\listoffigures

\listoftables

\newpage

\section{Introducción }

El problema que vamos a tratar de resolver desde la aproximación de un algoritmo Greedy consiste en minimizar el número de visitas que debe realizar un granjero a su proveedor, es decir, nos encontramos frente a un problema de optimización.
	
%----------------------------------------------------------------------------------------

\section{Resolución del problema}

Tenemos un granjero cuyo fertilizante dura R días. Además conocemos los días en los que abre la tienda.
Con estas características podemos tener dos posibilidades para resolver este problema:
\begin{itemize}
	\item Buscar la fecha mas cercana dentro del intervalo R
	\item Buscar la fecha mas lejana dentro del intervalor R
\end{itemize}

Como lo que queremos es optimizar, es decir, reducir el número de viajes que debe dar el granjero descartamos la primera opción y nos centramos en la segunda.

%------------------------------------------------------------------------------------------

\section{Elementos de la solución al problema}

\subsection{Conjunto de candidatos}
El conjunto de candiatos serán aquellos días en los que la tienda se encuentre abierta.

\subsection{Conjunto de seleccionados}
El conjunto de seleccionados serán los días elegidos para acudir a la tienda.

\subsection{Función solución}
Nuestra función solución será si han pasado todos esos días y no nos hemos quedado sin fertilizante, lo que significará que hemos maximizado el número de viajes.

\subsection{Función de factibilidad}
Sabremos que no hay solución, si dentro del intervalo R no hay ninguna tienda a la que acudir.

\subsection{Función selección}
Se seleccionará el día más lejano de los posibles.

\subsection{Función objetivo}
La solución devolverá una lista, con los dias en los que el granjero irá a  la tienda.

%--------------------------------------------------------------------------------------------

\section{Demostración que encuentra siempre la solución óptima}

Vamos a demostrar que encuentra siempre la solución óptima por una reducción al absurdo:\\

Sea $ L = g_0 < g_1 < ... < g_p $ el conjunto de días seleccionados por el algoritmo Greedy que \textbf{NO} es óptimo.\\
Sea $ L_{op} = f_0 < f_1 < ... < f_p $ una de la soluciones optimas del problema.\\
Sea $r$ el máximo valor posible hasta donde L y $L_{op}$ coinciden, es decir, $f_0=g:0, f_1=g_1,...,f_r = g_r$.\\
Entonces, $g(r+1)$ será mayor que $f(r+1)$ (ya que hemos supuesto que es en r cuando dejan de ser iguales y donde se inclina la balanza a la solución optima).\\
Por lo que $g_0 < ... < g_r < g(r+1) < f(r+2) < f_q $ es otra solucion al problema ya que  hemos llegado al final sin agotar el abono. Y ya que nuestro criterio para seleccionar es el día más lejano. Si f llega un momento que se separa de g (g es la nuestra). Quiere decir que una de dos. O selecciona más días (lo que ya no puede ser óptimo ya que la nuestra ha seleccionado menos dias) o f es igual que g (mínimo número de visitas a la tienda), por lo que tendría el mismo tamaño que $L_{op}$.\\
Por lo que alcanzamos una contradiccion, ya que $r$  no seria el maximo valor donde $L$ y $L_{op}$ siguen siendo iguales.



\section{Pseudocódigo}

\begin{verbatim}
Aquí va el pseudocódigo
\end{verbatim}

\end{document}