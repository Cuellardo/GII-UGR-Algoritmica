%%%%%%%%%%%%%%%%%%%%%%%%%%%%%%%%%%%%%%%%%
% Beamer Presentation
% LaTeX Template
% Version 1.0 (10/11/12)
%
% This template has been downloaded from:
% http://www.LaTeXTemplates.com
%
% License:
% CC BY-NC-SA 3.0 (http://creativecommons.org/licenses/by-nc-sa/3.0/)
%
%%%%%%%%%%%%%%%%%%%%%%%%%%%%%%%%%%%%%%%%%

%----------------------------------------------------------------------------------------
%	PACKAGES AND THEMES
%---------------------------------------------------------------------------------------

\documentclass{beamer}
\usepackage[spanish]{babel}
 \usepackage{algpseudocode}
\usepackage[utf8]{inputenc}
\mode<presentation> {


\usetheme{PaloAlto}

}

\usepackage{graphicx} % Allows including images
\usepackage{booktabs} % Allows the use of \toprule, \midrule and \bottomrule in tables

%----------------------------------------------------------------------------------------
%	TITLE PAGE
%----------------------------------------------------------------------------------------

\title[Practica 4]{Cena de gala} % The short title appears at the bottom of every slide, the full title is only on the title page

\author{Algorítmica} % Your name
\institute[UGR] % Your institution as it will appear on the bottom of every slide, may be shorthand to save space
{
Universidad de Granada \\ % Your institution for the title page
\medskip

}
\date{\today} % Date, can be changed to a custom date

\begin{document}

\begin{frame}
\titlepage % Print the title page as the first slide
\end{frame}

\begin{frame}
\frametitle{Índice} % Table of contents slide, comment this block out to remove it
\tableofcontents % Throughout your presentation, if you choose to use \section{} and \subsection{} commands, these will automatically be printed on this slide as an overview of your presentation
\end{frame}

%----------------------------------------------------------------------------------------
%	PRESENTATION SLIDES
%----------------------------------------------------------------------------------------

\section{Introducción }
\begin{frame}
	\frametitle{Introducción}
	\begin{itemize}
		\item El objetivo de esta práctica es diseñar un algoritmo Bactraking, que resuelva uno de los cinco problemas de la práctica y realizar un estudio empírico de su eficiencia.
	\end{itemize}
\end{frame}


%------------------------------------------------
\section{Ejercicio} 
\begin{frame}
	\frametitle{Enunciado del ejercicio}
	Se desea sentar a N invitados alrededor de una mesa, de manera que cada invitado tendra a su lado a otros dos. Cada par de invitados tiene un nivel de compatibilidad. Se desea maximizar la compatibilidad de estos comensales.
	
\end{frame}

%------------------------------------------------
\section{Diseño del algoritmo} 
\begin{frame}
	\frametitle{Diseño del algoritmo}
	\begin{itemize}
		\item \textbf{Solución parcial}: Solucion parcial al problema de tamaño menor que N. (Conjunto \textbf{Sp})
		\item \textbf{Funcion de poda}: No se me ocurre nada.
		\item \textbf{Restricciones explícitas}: Los valores que puede tomar la solucion son los enteros de 1 a N. Donde N es el número total de invitados.  
		\item \textbf{Restricciones implícitas}: Estas restricciones son las que determinan si una función parcial puede llevarnos a una solucion del problema. Si supera un umbral.  
	\end{itemize}
	
\end{frame}

\section{Pseudocodigo}
\begin{frame}

	
\end{frame}







\end{document} 
